\documentclass[11pt,a4paper]{article}
\usepackage[utf8]{inputenc} 
\usepackage[T1]{fontenc} % fonts to encode unicode
\usepackage{times}
\usepackage{url}
\sloppy
\hyphenpenalty 10000

\setlength\topmargin{-5mm} \setlength\oddsidemargin{-0cm}
\setlength\textheight{24.7cm} \setlength\textwidth{16cm}
\setlength\columnsep{0.6cm}  \newlength\titlebox \setlength\titlebox{2.00in}
\setlength\headheight{5pt}   \setlength\headsep{0pt}
\setlength\footskip{1.0cm}
\setlength\leftmargin{0.0in}
\pagestyle{empty}

\setlength{\parindent}{0in}
\setlength{\parskip}{2ex}

\begin{document}

\begin{center}
  {\Large \bf Preface}
\end{center}

\vspace*{0.5cm}

CNL 2020/21 is the seventh edition of the workshop series on Controlled Natural Language (CNL)\footnote{\url{http://www.sigcnl.org/cnl2020.html}}. It was initially planned for 2020, but had to be postponed by one year due to the Covid-19 pandemic. It is co-located with the SEMANTiCS 2021 conference\footnote{\url{https://2021-eu.semantics.cc/}} and will be held on 8 and 9 September 2021 in Amsterdam, as a hybrid event where onsite as well as online participation is possible.

We received 15 full paper submissions and two submissions as short papers. These 17 papers then received in total 54 reviews from the members of the program committee, which corresponds to an average of 3.2 reviews per paper. Out of the full paper submissions, eleven were accepted as such. Of the remaining four, three were rejected as full papers but accepted as short ones. The two short paper submissions were both accepted. Therefore, these proceedings include eleven full papers and five short ones.

We would like to thanks all authors for their submissions and the program committee members for their careful reviews. We are now looking forward to the workshop with this exciting program.

Tobias Kuhn\\
Silvie Spreeuwenberg\\
Stijn Hoppenbrouwers\\
Norbert E. Fuchs\\
 
\end{document}
